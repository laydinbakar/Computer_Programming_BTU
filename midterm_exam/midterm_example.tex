%   Author        : Levent Aydinbakar
%   Date          : 20-58-04---19-11-2022
%   Last Modified : 02-34-38---20-11-2022


\documentclass[addpoints]{exam}

\usepackage[a4paper, total={18cm, 25cm}]{geometry} % For setting the PDF file page sizes

\usepackage{multicol}
\usepackage[utf8]{inputenc}
\usepackage[english]{babel}
\usepackage{lastpage}
\usepackage{graphicx}
\usepackage{xcolor}
\usepackage{caption}
\usepackage{amsmath, mathtools}
\usepackage{paralist}
\usepackage{listings} % For writing code examples 
\lstset{basicstyle=\ttfamily,
  showstringspaces=false,
  commentstyle=\color{gray},
  keywordstyle=\color{blue},
}
\renewcommand{\lstlistingname}{Script}% Listing -> Algorithm


\renewcommand{\thequestion}{\bfseries{Q\arabic{question}}}
\renewcommand{\thepartno}{\bfseries{\Alph{partno}}}


\rhead{}
\rfoot{Page \thepage \hspace{1pt} of \pageref{LastPage}}
\cfoot{}
\begin{document}
\begin{center}
%\begin{figure}[h!]
%\includegraphics[width=2cm]{btu_logo.jpg} 
\includegraphics[%
    width=3cm,
%    align=t,
%    smash=br,
%    vshift=1cm,     % adjust the vertical position
%    hshift=-1.5cm     % adjust the horizontal position
]{figures/btu_logo}%
%\caption{}
%\end{figure}

\begin{large}

\textbf{Computer Programming (C/C++) MECH0291}

\textbf{
\begin{LARGE}
\textcolor{red}{Example}
Midterm Exam 
\end{LARGE}}

\vspace{0.2cm}
\begin{LARGE}
\textbf{Group 1/A}
\end{LARGE}

\vspace{0.2cm}
November 25, 2022, 17:00--21:00

\vspace{0.2cm}


\vspace{0.2cm}
\end{large}
\end{center}
%\begin{multicols}{2}
\begin{minipage}[c]{0.45\linewidth}
\begin{flushleft}
\begin{tabular}{l l l }
Student Name	&:   & \hspace{1cm} \\
Student ID		&:   & \hspace{1cm} \\
Department		&:   & \hspace{1cm} \\
  				& 	 & \hspace{1cm} \\
Instructor		&:   & Dr. Levent Aydinbakar\\
Teaching Assistant		&:   & RA Ismail Hos\\
\end{tabular}
\end{flushleft}

\end{minipage}
%\columnbreak
\begin{minipage}[c]{0.6\linewidth}
\begin{flushleft}
\gradetable[h]
\end{flushleft}
%\end{multicols}
\end{minipage}

\begin{flushleft}

\fbox{\fbox{\parbox{17.5cm}{

\textbf{Instructions}
\begin{enumerate}
    \item This booklet contains \pageref{LastPage} pages.
    \item You have 75~minutes to complete the examination.
	\item You may \textbf{only} use the Terminal Application in this exam. You \textbf{may not} use any web browser or any other programs.
	\item Only students who has an appropriate GitLab Repository can join this exam. Your \$USER\_NAME should be in\ \texttt{yourStudentID\_name\_surname} format.
	\item Start with the commands below to set the computer that you use GitLab on the exam computer.
	\begin{itemize}
		\item \texttt{git config --global user.name "\$USER\_NAME"}
		\item \texttt{git config --global user.email "\$USER\_EMAIL"}
	\end{itemize}
	\item Remove the old \texttt{ComputerProgramming2022} folder if exists on your working directory.
	\item Download your ComputerProgramming2022 repository from GitLab with the command below, create a folder as \texttt{MidtermExam} if you do not have already created, and put the files you make in this exam in this folder. 
	\begin{itemize}
		\item \texttt{git clone https://gitlab.com/\$USER\_NAME/ComputerProgramming2022.git}
	\end{itemize}
	\item At the end, you must upload your codes into the same repository using the commands below. Those files will be evaluated and marked as your midterm exam grade.
	\begin{itemize}
		\item \texttt{git add MidtermExam}
		\item \texttt{git commit -m "Add MidterExam"}
		\item \texttt{git push origin main}
	\end{itemize}
	\item Add your student ID, name and surname as a comment on the top of each scripts you write in this exam.
    \item You may use one (1) double-sided A4 paper ($210\times297~{\mathrm{mm}^2}$) with notes that you have prepared in your handwriting. You may not use printed or photocopied paper sheets, lecture notes, books, or other students.
    \item The maximum point you can obtain in this exam is 100. 
\end{enumerate}}}}

\end{flushleft}
\begin{questions}
\newpage
\question[25]  

Write a shell script (\texttt{shell1.sh}) in \texttt{MidtermExam} folder as described below.
\begin{itemize}
\item Create a folder so called \texttt{shell1}. Change \texttt{shell1.sh}'s working directory into it.
\item Make 3 folders with the names \texttt{1}, \texttt{2} and \texttt{3} in \texttt{shell1}.
\item Make 11 folders with the names $1$ to $11$ in each $3$ directories (\texttt{1}, \texttt{2} and \texttt{3}). The folder names should have four characters, filled with zeros such as (\texttt{0005} and \texttt{0010}). 
\item Generate 101 text files with the names $1$ to $101$ in each $3\times11$ folders. File names should have four characters except the file extension (e.g. \texttt{0001.txt}).
\item Write the relative path to your working directory (\texttt{MidtermExam}) into each file as ``This is $n$th file in \$RELATIVE\_PATH.". Here $n$ is the file number from 1 to 101 without zeros.
\item Make the shell script executable.
\end{itemize}


\question[25]

Write a shell script (\texttt{shell2.sh}) in \texttt{MidtermExam} folder as described below.
\begin{itemize}
\item Create a directory so called \texttt{shell2}. Do not change \texttt{shell2.sh}'s working directory into it.
\item Make 3 directories in \texttt{shell2}. Read the names from the commandline. Name the folders as \texttt{F1}, \texttt{F2} and \texttt{F3}.
\item Make 13 directories with the names $1$ to $13$ in each $3$ directories (\texttt{F1}, \texttt{F2} and \texttt{F3}). The folder names should have four characters, filled with zeros such as (\texttt{0005} and \texttt{0010}). 
\item Generate 20 text files with the names $20$ to $115$ skipping $5$ in each $3\times13$ folders. File names should have three characters except the file extension (e.g. \texttt{020.txt}).
\item Write in the files if it is an odd number file or an even number file. Such as, ``This is an odd file.'' in \texttt{020.txt}.
\item Make the shell script executable.
\end{itemize}


\question[25]

Write a Python script (\texttt{python1.sh}) in \texttt{MidtermExam} as described below.
\begin{itemize}
\item Make a folder as \texttt{python\_output}.
\item Create a $3\times1000$ array of ones.
\item Multiply each element of the array with the column number, row number and $\pi$ number.
\item Write the array into a text file and a binary file, both located in \texttt{python\_output} folder.
\item Check the sizes of the files and print as, for example ``Size of text.txt file is 10MB and binary file 10MB.''.
\item Make the Python script executable.
\end{itemize}



\question[25]

Write a Python script (\texttt{python2.sh}) in \texttt{MidtermExam} as described below.
\begin{itemize}
\item Use argument parser module to read
\begin{itemize}
\item an output file name, and
\item two folder names from the commandline.
\end{itemize}
\item Make two folders with the names \texttt{full\_ones} and \texttt{ones\_with\_fives} in \texttt{python\_output} folder. Read these names as argument.
\item Create a $2\times50$ array of ones.
\item Write the array into a binary file with the name you read as an argument, in \texttt{full\_ones} folder.
\item Multiply the both items of each line by five, if the line number is a power of three.
\item Write the new array into a binary file with the name you read as an argument, in \texttt{ones\_with\_fives} folder.
\item Make the Python script executable.
\end{itemize}


\question[10]

Add a \texttt{README.md} file briefly explaining what does each script do in this repository.

\question[10]

Write two shell scripts (\texttt{run\_all.sh} and \texttt{remove.sh}) in \texttt{MidtermExam} folder as described below.
\begin{itemize}
\item \texttt{run\_all.sh} to run all the scripts in this directory, and
\item \texttt{run\_all.sh} to rint your student id, name, surname and the total points you expect to get from this exam.
\item \texttt{remove.sh} to remove all the files and folders you made in this exam, \textbf{except}
\begin{itemize}
\item\texttt{shell1.sh},
\item\texttt{shell2.sh},
\item\texttt{python1.py},
\item\texttt{python2.py},
\item\texttt{README.md},
\item\texttt{run\_all.sh}, and
\item\texttt{remove.sh}.
\end{itemize}
\end{itemize}



\end{questions}
\end{document}


