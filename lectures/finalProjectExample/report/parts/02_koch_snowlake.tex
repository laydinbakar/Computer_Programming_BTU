\clearpage
\newpage
\mbox{~}
\section{Koch Snowflake}
\label{sec:KochSnowflake}

In Section~\ref{sec:SierpinskiTriangle}, we show how to make the Sierpinski Triangle.
Here we will show the Koch Snowflake \cite{KS, OP, LA}.

The Koch Snowflake can also be obtained from a equilateral triangle.

\begin{figure}[htb]
\centering
\begin{tikzpicture}
\coordinate (A) at (0,0); 
\coordinate (B) at (4,0); 
\coordinate (C) at (2,3.464);
\draw[blue!50!white] (A) -- (B) -- (C) -- (A);
\end{tikzpicture}
\caption{An equilateral triangle corner points}
\label{fig:triangleKSP}
\end{figure}

Then the lines are divided into 3 equal parts, the middle part is rotated $60^\circ$ around both of the points.

\begin{figure}[htb]
\centering
\begin{tikzpicture}
\coordinate (A) at (0,0); 
\coordinate (B) at (4,0); 
\coordinate (C) at (2,3.464);
\draw[blue!50!white] (A) -- (B) -- (C) -- (A);
\node at (2,1.15){\includegraphics{KSP1.pdf}};
\end{tikzpicture}
\caption{A first order Koch Snowflake}
\label{fig:KSP1}
\end{figure}

When we continue dividing each line and rotating the midline, we can obtain Figure~\ref{fig:KSP2}.

\textcolor{gray}{
\lipsum[2]
}

\begin{figure}[htb]
\centering
\begin{tikzpicture}
\node[anchor=north west] at (0,4.8)  {\includegraphics[page=1]{KSP2.pdf}};
\node[anchor=north west] at (0,0)    {\includegraphics[page=2]{KSP2.pdf}};
\node[anchor=north west] at (0,-4.8) {\includegraphics[page=3]{KSP2.pdf}};
\node[anchor=north west] at (0,-9.6) {\includegraphics[page=4]{KSP2.pdf}};
\end{tikzpicture}
\caption{A second, third, fourth, fifth, and sixth order Koch Snowflakes from 
from \textit{top} to \textit{bottom}}
\label{fig:KSP2}
\end{figure}

%The length of the equilateral triangle given in Figure~\ref{fig:triangleKSP} is $l$. 
%The perimeter of the triangle is defined as $3l$.
%When the rule shown in Figure~\ref{fig:KSP1} is applied, the new figure has $\frac{4}{3}l$.
%Then the perimeter of the new shape is $3\frac{4}{3}l$.
%The perimeter of the second order shape is calculated as $3\frac{4}{3}^2l$.
%When we apply the rule $n$ times, then the perimeter is defined as $3\frac{4}{3}^nl$.

